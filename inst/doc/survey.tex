\documentclass{article}
\usepackage{url}
%\VignetteIndexEntry{A survey analysis example}
\usepackage{Sweave}
\author{Thomas Lumley}
\title{A survey analysis example}

\begin{document}
\maketitle

This document provides a simple example analysis of a survey data set,
a subsample from the California Academic Performance Index, an annual
set of tests used to evaluate California schools. The API website,
including the original data files are at
\url{http://api.cde.ca.gov}. The subsample was generated as a teaching
example by Academic Technology Services at UCLA and was obtained from
\url{http://www.ats.ucla.edu/stat/stata/Library/svy_survey.htm}.


We have a cluster sample in which 15 school districts were sampled and
then all schools in each district. This is in the data frame
\texttt{apiclus1}, loaded with \texttt{data(api)}. The two-stage sample is
defined by the sampling unit (\texttt{dnum}) and the population
size(\texttt{fpc}).  Sampling weights are computed from the population
sizes, but could be provided separately.
\begin{Schunk}
\begin{Sinput}
> data(api)
> dclus1 <- svydesign(id = ~dnum, weights = ~pw, data = apiclus1, 
+     fpc = ~fpc)
\end{Sinput}
\end{Schunk}

The \texttt{svydesign} function returns an object containing the survey data and metadata.
\begin{Schunk}
\begin{Sinput}
> summary(dclus1)
\end{Sinput}
\begin{Soutput}
1 - level Cluster Sampling design
With (15) clusters.
svydesign(id = ~dnum, weights = ~pw, data = apiclus1, fpc = ~fpc)
Probabilities:
   Min. 1st Qu.  Median    Mean 3rd Qu.    Max. 
0.02954 0.02954 0.02954 0.02954 0.02954 0.02954 
Population size (PSUs): 757 
Data variables:
 [1] "cds"      "stype"    "name"     "sname"    "snum"     "dname"   
 [7] "dnum"     "cname"    "cnum"     "flag"     "pcttest"  "api00"   
[13] "api99"    "target"   "growth"   "sch.wide" "comp.imp" "both"    
[19] "awards"   "meals"    "ell"      "yr.rnd"   "mobility" "acs.k3"  
[25] "acs.46"   "acs.core" "pct.resp" "not.hsg"  "hsg"      "some.col"
[31] "col.grad" "grad.sch" "avg.ed"   "full"     "emer"     "enroll"  
[37] "api.stu"  "fpc"      "pw"      
\end{Soutput}
\end{Schunk}

We can compute summary statistics to estimate the mean, median, and
quartiles of the Academic Performance Index in the year 2000, the
number of elementary, middle, and high schools in the state, the total
number of students, and the proportion who took the test.  Each
function takes a formula object describing the variables and a survey
design object containing the data.
\begin{Schunk}
\begin{Sinput}
> svymean(~api00, dclus1)
\end{Sinput}
\begin{Soutput}
        mean     SE
api00 644.17 23.542
\end{Soutput}
\begin{Sinput}
> svyquantile(~api00, dclus1, quantile = c(0.25, 0.5, 0.75), ci = TRUE)
\end{Sinput}
\begin{Soutput}
$quantiles
        0.25 0.5  0.75
api00 551.75 652 717.5

$CIs
, , api00

           0.25      0.5     0.75
(lower 493.2835 564.3250 696.0000
upper) 622.6495 710.8375 761.1355
\end{Soutput}
\begin{Sinput}
> svytotal(~stype, dclus1)
\end{Sinput}
\begin{Soutput}
         total      SE
stypeE 4873.97 1333.32
stypeH  473.86  158.70
stypeM  846.17  167.55
\end{Soutput}
\begin{Sinput}
> svytotal(~enroll, dclus1)
\end{Sinput}
\begin{Soutput}
         total     SE
enroll 3404940 932235
\end{Soutput}
\begin{Sinput}
> svyratio(~api.stu, ~enroll, dclus1)
\end{Sinput}
\begin{Soutput}
Ratio estimator: svyratio.survey.design2(~api.stu, ~enroll, dclus1)
Ratios=
           enroll
api.stu 0.8497087
SEs=
             enroll
api.stu 0.008386297
\end{Soutput}
\end{Schunk}

The ordinary R subsetting functions \verb'[' and \texttt{subset} work
correctly on these survey objects, carrying along the metadata needed
for valid standard errors. Here we compute the proportion of high
school students who took the test
\begin{Schunk}
\begin{Sinput}
> svyratio(~api.stu, ~enroll, design = subset(dclus1, stype == 
+     "H"))
\end{Sinput}
\begin{Soutput}
Ratio estimator: svyratio.survey.design2(~api.stu, ~enroll, design = subset(dclus1, 
    stype == "H"))
Ratios=
           enroll
api.stu 0.8300683
SEs=
            enroll
api.stu 0.01472607
\end{Soutput}
\end{Schunk}

The warnings referred to in the output occured because several
school districts have only one high school sampled, making the second
stage standard error estimation unreliable.

Specifying a large number of variables is made easier by the \texttt{make.formula} function
\begin{Schunk}
\begin{Sinput}
> vars <- names(apiclus1)[c(12:13, 16:23, 27:37)]
> svymean(make.formula(vars), dclus1, na.rm = TRUE)
\end{Sinput}
\begin{Soutput}
                  mean      SE
api00       643.203822 25.4936
api99       605.490446 25.4987
sch.wideNo    0.127389  0.0247
sch.wideYes   0.872611  0.0247
comp.impNo    0.273885  0.0365
comp.impYes   0.726115  0.0365
bothNo        0.273885  0.0365
bothYes       0.726115  0.0365
awardsNo      0.292994  0.0397
awardsYes     0.707006  0.0397
meals        50.636943  6.6588
ell          26.891720  2.1567
yr.rndNo      0.942675  0.0358
yr.rndYes     0.057325  0.0358
mobility     17.719745  1.4555
pct.resp     67.171975  9.6553
not.hsg      23.082803  3.1976
hsg          24.847134  1.1167
some.col     25.210191  1.4709
col.grad     20.611465  1.7305
grad.sch      6.229299  1.5361
avg.ed        2.621529  0.1054
full         87.127389  2.1624
emer         10.968153  1.7612
enroll      573.713376 46.5959
api.stu     487.318471 41.4182
\end{Soutput}
\end{Schunk}

Summary statistics for subsets can also be computed with
\texttt{svyby}. Here we compute the average proportion of ``English
language learners'' and of students eligible for subsidized school
meals for elementary, middle, and high schools
\begin{Schunk}
\begin{Sinput}
> svyby(~ell + meals, ~stype, design = dclus1, svymean)
\end{Sinput}
\begin{Soutput}
  stype statistics.ell statistics.meals   se.ell se.meals
E     E       29.69444         53.09028 1.411617 7.070399
H     H       15.00000         37.57143 5.347065 5.912262
M     M       22.68000         43.08000 2.952862 6.017110
\end{Soutput}
\end{Schunk}


Regression models show that these socieconomic variables predict API score and whether the school achieved its API target
\begin{Schunk}
\begin{Sinput}
> regmodel <- svyglm(api00 ~ ell + meals, design = dclus1)
> logitmodel <- svyglm(I(sch.wide == "Yes") ~ ell + meals, design = dclus1, 
+     family = quasibinomial())
> summary(regmodel)
\end{Sinput}
\begin{Soutput}
Call:
svyglm(api00 ~ ell + meals, design = dclus1)

Survey design:
svydesign(id = ~dnum, weights = ~pw, data = apiclus1, fpc = ~fpc)

Coefficients:
            Estimate Std. Error t value Pr(>|t|)    
(Intercept) 817.1823    18.6709  43.768 1.32e-14 ***
ell          -0.5088     0.3259  -1.561    0.144    
meals        -3.1456     0.3018 -10.423 2.29e-07 ***
---
Signif. codes:  0 ‘***’ 0.001 ‘**’ 0.01 ‘*’ 0.05 ‘.’ 0.1 ‘ ’ 1 

(Dispersion parameter for gaussian family taken to be 3161.207)

Number of Fisher Scoring iterations: 2
\end{Soutput}
\begin{Sinput}
> summary(logitmodel)
\end{Sinput}
\begin{Soutput}
Call:
svyglm(I(sch.wide == "Yes") ~ ell + meals, design = dclus1, family = quasibinomial())

Survey design:
svydesign(id = ~dnum, weights = ~pw, data = apiclus1, fpc = ~fpc)

Coefficients:
             Estimate Std. Error t value Pr(>|t|)   
(Intercept)  1.899557   0.509900   3.725  0.00290 **
ell          0.039925   0.012442   3.209  0.00751 **
meals       -0.019115   0.008825  -2.166  0.05116 . 
---
Signif. codes:  0 ‘***’ 0.001 ‘**’ 0.01 ‘*’ 0.05 ‘.’ 0.1 ‘ ’ 1 

(Dispersion parameter for quasibinomial family taken to be 0.9627734)

Number of Fisher Scoring iterations: 5
\end{Soutput}
\end{Schunk}

We can calibrate the sampling using the statewide total for the previous year's API 
\begin{Schunk}
\begin{Sinput}
> gclus1 <- calibrate(dclus1, formula = ~api99, population = c(6194, 
+     3914069))
\end{Sinput}
\end{Schunk}
which improves estimation of some quantities
\begin{Schunk}
\begin{Sinput}
> svymean(~api00, gclus1)
\end{Sinput}
\begin{Soutput}
        mean     SE
api00 666.72 3.2959
\end{Soutput}
\begin{Sinput}
> svyquantile(~api00, gclus1, quantile = c(0.25, 0.5, 0.75), ci = TRUE)
\end{Sinput}
\begin{Soutput}
$quantiles
          0.25     0.5     0.75
api00 592.0652 681.181 736.5414

$CIs
, , api00

           0.25      0.5     0.75
(lower 553.0472 663.3175 721.1432
upper) 617.0915 696.8528 755.2959
\end{Soutput}
\begin{Sinput}
> svytotal(~stype, gclus1)
\end{Sinput}
\begin{Soutput}
         total     SE
stypeE 4881.77 302.15
stypeH  463.35 183.03
stypeM  848.88 194.76
\end{Soutput}
\begin{Sinput}
> svytotal(~enroll, gclus1)
\end{Sinput}
\begin{Soutput}
         total     SE
enroll 3357372 243227
\end{Soutput}
\begin{Sinput}
> svyratio(~api.stu, ~enroll, gclus1)
\end{Sinput}
\begin{Soutput}
Ratio estimator: svyratio.survey.design2(~api.stu, ~enroll, gclus1)
Ratios=
           enroll
api.stu 0.8506941
SEs=
             enroll
api.stu 0.008674888
\end{Soutput}
\end{Schunk}


\end{document}
